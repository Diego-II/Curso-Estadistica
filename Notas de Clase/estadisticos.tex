%!TEX root = apunte_estadistica.tex


\chapter{Estadísticos}


\clase{Clase 4: 13 de agosto}
\section{Estadísticos}

Un estadístico es una función de (los valores de) una variable aleatoria, definida desde el espacio muestral. 

\begin{definition}[Estadístico]
\label{def:estadístico}
Sea $(S,\cA,\mu)$ un espacio de probabilidad y $X\in\cX$ una variable aleatoria con distribución paramétrica $\cP = \{P_\theta\ \tq\ \theta\in\Theta\}$. Un estadístico es una función medible de $X$ independiente del parámetro $\theta$.
\begin{align}
	T:\ &\cX \rightarrow \cT\\
	&x\mapsto T(x)
\end{align} 

\end{definition}


Es importante diferenciar el valor particular que toma $T(x)$, cuando $X$ toma el valor específico $X=x$, de la variable aleatoria resultante de la aplicación de la función $T(\cdot)$ a la variable aleatoria $X$, es decir, $T(X)$. Este último tiene su propia distribución de probabilidad inducida por $X$ y por la función $T$ propiamente tal. 

Algunos estimadores pueden ser: 
\begin{equation}
	T(x) = \frac{1}{n}\sum_{i=1}^nx_i,\qquad T'(x) = x, \qquad T''(x) = \min(x).
\end{equation}
En términos generales, el objetivo de un estadístico es \textit{encapsular} o \textit{resumir} la información contenida en una muestra de datos $x = (x_1,x_2,\ldots,x_n)$ que es de utilidad para determinar (o estimar) el parámetro de la distribución de $X$. Por esta razón, la función identidad o el promedio parecen cumplir, al menos intuitivamente, con esta misión. No así $T''$ en el ajemplo anterior. 

Para formalizar esta idea, consideremos la siguiente definición


\begin{definition}[Estadístico Suficiente]
\label{def:estadístico_suficiente}
Sea $(S,\cA,\mu)$ un espacio de probabilidad y $X\in\cX$ una variable aleatoria con distribución paramétrica $\cP = \{P_\theta\ \tq\ \theta\in\Theta\}$. Diremos que la función $T:\cX\rightarrow\cT$ es un estadístico suficiente para $\theta$ (o para $X$ o para $\cP$) si la ley condicional $X|T(X)$ no depende del parámetro $\theta$, es decir, 
\begin{align}
	P_\theta(X\in A | T(X)),\ A\in\cB(X), \text{no depende de }\theta.
\end{align} 
\end{definition}

Observemos entonces que si $T(X)$ es un estadístico suficiente, entonces, existe una función 

\begin{equation}
	H(\cdot,\cdot): \cB(X)\times\cT \rightarrow [0,1]
\end{equation}
que es una distribución de probabilidad en el primer argumento y es medible en el segundo argumento. 
fg
\begin{example}[Estadístico suficiente trivial]
	\label{ex:suficiencia_trivial}
	Para cualquier familia paramétrica $\cP$, el estadístico definido por
	\begin{equation}
		T(x) = x
	\end{equation}
es suficiente. En efecto, $P_\theta(X\in A|X=x) = \ind_{A}(x)$ no depende del parámetro de la familia. 
\end{example}

\begin{example}[Estadístico suficiente Bernoulli]
	Sea $x=(x_1,\ldots,x_n) \sim Ber(\theta)$, $\theta \in \Theta = [0,1]$, es decir
	\begin{equation}
		P_\theta(X=x) = \theta^{\sum x_i}(1-\theta)^{n-\sum x_i}.
	\end{equation}
	Veamos que $T(x) = \sum x_i$ es un estadístico suficiente (por definición). En efecto
	\begin{alignat*}{3}
		P(X=x|T(X)=t) 	&= \frac{P(T(X)=t| X=x )P( X=x )}{P(T(X)=t)} \quad&&\text{(T. Bayes)}\\
						&= \frac{\ind_{T(x)=t}\theta^{\sum x_i}(1-\theta)^{n-\sum x_i}}{\binom{n}{t}\theta^t(1-\theta)^{n-t}} &&\text{(reemplazando modelo)}\\
						&= \binom{n}{t}^{-1} && \text{(pues $T(x)=t$)}
	\end{alignat*}
	Consecuentemente, $T(x)=\sum x_i$ es estadístico suficiente.
\end{example}

Intuitivamente, nos gustaría poder verificar directamente de la suficiencia de un estadístico desde la distribución o densidad de una VA, o al menos verificar una condición más simple que la definición. Esto es porque verificar la no-dependencia de la distribución condicional $P(X|T)$ puede ser no trivial, engorroso o tedioso. Para esto enunciaremos el Teorema de Fisher-Neyman, el cual primero requiere revisar la siguiente definición. 



\begin{definition}[Familia Dominada]
	Una familia de modelos paramétricos $\familiaparametrica$ es dominada si existe una medida $\mu$, tal que $\forall \theta\in\Theta, P_\theta$ es absolutamente continua con respecto a $\mu$ (denotado $ P_\theta \ll \mu$), es decir, 

	\begin{equation}
		\forall \theta\in\Theta, A\in\cB(X), \mu(A)=0 \Rightarrow P_\theta(A) = 0 
	\end{equation}
\end{definition}

La definición anterior puede interpretarse de la siguiente forma: si una familia de modelos paramétricos es dominada por una medida $\mu$, entonces ninguno de sus elementos puede asignar medida (probabilidad) no nula a conjuntos que tienen medida cero bajo $\mu$ (la medida \textit{dominante}). Una consecuencia fundamental de que la distribución $P_\theta$ esté dominada por $\mu$ está dada por el Teorema de Radon–Nikodym,  el cual establece que si $ P_\theta \ll \mu$, entonces la distribución $P_\theta$ tiene una densidad, es decir,	
	\begin{equation}
		\forall A\in\cB(X), P_\theta(X\in A) = \int _A p_\theta(x) \mu(\d x)
	\end{equation}
donde $p_\theta(x)$ es conocida como la densidad de $P_\theta$ con respecto a $\theta$ (o también como la derivada de Radon–Nikodym  $\frac{d P_\theta}{d \mu}$).

Con la noción de Familia Dominada y de densidad de probabilidad, podemos enunciar el siguiente teorema que conecta la forma de la densidad de un modelo paramétrico con la suficiencia de su estadístico. 

\clase{Clase 5: 20 de agosto}
\begin{theorem}[Factorización, Neyman-Fisher]
	\label{teo:neyman-fisher}
	Sea $\familiaparametrica$  una familia dominada por $\mu$, entonces, $T$ es un estadístico suficiente si y solo si existen funciones apropiadas $g_\theta(\cdot)$ y $h(\cdot)$, i.e., medibles y no-negativas, tal que la densidad de las distribuciones en $\cP$ se admiten la factorización  
	\begin{equation}
		\label{eq:neyman-fisher}
		p_\theta (x) = g_\theta(T(x))h(x)  
	\end{equation}
\end{theorem}

El Teorema de Neyman-Fisher es clave para evaluar, directamente de la densidad de un modelo, la suficiente de un estadístico. Pues al identificar la expresión de la VA que interactúa con el parámetro (en la función $g_\theta$) es posible determinar el estadístico suficiente. Antes de ver una demostración informal del Teorema \ref{teo:neyman-fisher}, revisemos un par de ejemplos.

\begin{example}[Factorización Bernoulli]
	Notemos que la densidad de Bernoulli (que es igual a su distribución por ser un modelo discreto) factoriza tal como se describe en el Teorema \ref{teo:neyman-fisher}. En efecto, consideremos $x=(x_1,\ldots, x_n)\sim$ Bernoulli($\theta$) y el estadístico $T(x) = \sum x_i$, entonces, 
	\begin{equation}
		p(X=x) = \underbrace{\theta^{\sum x_i}(1-\theta)^{n-\sum x_i}}_{g_\theta(T(x))} \cdot \underbrace{1}_{h(x)}
	\end{equation}
\end{example}

\begin{example}[Factorización Normal (varianza conocida)]
	Consideremos ahora $x=(x_1,\ldots, x_n)\sim$ $\cN(\mu,\sigma^2$), con $\sigma^2$ conocido y el estadístico $T(x) = \frac{1}{n}\sum x_i$, entonces, 
	\begin{align*}
		p(X=x) & = \prod_{i=1}^n \frac{1}{\sqrt{2\pi\sigma^2}}\exp\left(-\frac{1}{2\sigma^2}(x_i-\mu)^2\right)\\
		&=  (2\pi\sigma^2)^{-n/2}\exp\left(-\frac{1}{2\sigma^2}\sum_{i=1}^n(x_i-\mu)^2\right)\\
		&=  (2\pi\sigma^2)^{-n/2}\exp\left(-\frac{1}{2\sigma^2}\sum_{i=1}^n((x_i-\bar{x}) + (\bar{x}-\mu))^2\right)\\
		&=  (2\pi\sigma^2)^{-n/2}\exp\left(-\frac{1}{2\sigma^2}\sum_{i=1}^n (x_i-\bar{x})^2 + 2\cancel{(x_i-\bar{x})}(\bar{x}-\mu) + (\bar{x}-\mu)^2\right)\\
		&=  \underbrace{(2\pi\sigma^2)^{-n/2}\exp\left(-\frac{1}{2\sigma^2}\sum_{i=1}^n (x_i-\bar{x})^2\right)}_{h(x)} \underbrace{\exp\left( -\frac{1}{2\sigma^2}\sum_{i=1}^n (\bar{x}-\mu)^2\right)}_{g_\theta(T(x))}
	\end{align*}
\end{example}

A continuación, veremos la prueba del Teorema \ref{teo:neyman-fisher} para el caso discreto. 


\begin{proof}[Demostración de Teorema Neyman-Fisher, caso discreto]
Primero probamos la implicancia hacia la derecha ($\Rightarrow$), es decir, asumiendo que $T(X)$ es un estadístico suficiente, tenemos,
	\begin{alignat*}{3}
		p_\theta(X=x) 	&= P_\theta(X=x, T(X)=T(x))\\
						&= \underbrace{P_\theta(X=x| T(X)=T(x))}_{h(x)\text{, no depende de $\theta$ por hipótesis}} \underbrace{P_\theta(T(X)=T(x))}_{g_\theta(T(x))}
	\end{alignat*}
	es decir, la factorización deseada.

	Ahora probamos la implicancia hacia la izquierda ($\Leftarrow$), es decir, asumiendo la factorización en la ecuación \eqref{eq:neyman-fisher}, tenemos que el modelo se puede escribir como 

	\begin{equation*}
		\label{eq:bayes_NF}
		p_\theta(X=x|T(X)=t)=  \frac{p_\theta(T(X)=t|X=x)p_\theta(X=x)}{p_\theta(T(X)=t)}
	\end{equation*}
	Donde $p_\theta(T(X)=t|X=x)= \ind_{T(x)=t}$ y la hipótesis nos permite escribir 
	\begin{alignat*}{3}
		p_\theta(X=x)&=  g_\theta(T(x))h(x)\\
		p_\theta(T(X)=t) &= \sum_{x';T(x')=t}p_\theta(X=x') = \sum_{x';T(x')=t}g_\theta(T(x'))h(x')
	\end{alignat*}

	Incluyendo estas últimas dos expresiones en eq.\eqref{eq:bayes_NF}, tenemos 
	\begin{equation}
		\label{eq:NF_final}
		p_\theta(X=x|T(X)=t)=  \frac{\ind_{T(x)=t}\cancel{g_\theta(T(x))}h(x)}{\sum_{x';T(x')=t}\cancel{g_\theta(T(x'))}h(x')}=  \frac{\ind_{T(x)=t}h(x)}{\sum_{x';T(x')=t}h(x')}
	\end{equation}
	donde los términos que se cancelan son todos iguales a $g_\theta(t)$.

	Finalmente, como el lado derecho de la ecuación \eqref{eq:NF_final} no depende de $\theta$, se concluye la demostración.
\end{proof}

La idea de suficiencia del estadístico dice relación, coloquialmente, con la \textit{información} contenida en el estadístico que permite \textit{descubrir} el parámetro $\theta$. En ese sentido, se tiene la intuición que un estadístico es suficiente si tiene la información \textit{suficiente}. En el extremo de esta intuición, el estadístico puede ser simplemente todos los datos, i.e, $T(X)=X$, en cuyo caso la suficiencia es directa como se vio en el Ejemplo \ref{ex:suficiencia_trivial}, sin embargo, estaremos interesado en estadísticos que son suficientes pero que contienen la mínima cantidad de información. 

Sin una definición formal de \textit{información} aún, recordemos que los estadísticos representan un resumen o una compresión  de los datos mediante una función, i.e., la función $T(\cdot)$. Usando el mismo concepto, en el cual la aplicación de una función \textit{quita información desde la preimagen a la imagen}, podemos definir el siguiente concepto. 

\begin{definition}[Estadístico Suficiente Minimal]
	Un estadístico $T:\cX\rightarrow\cT$ es suficiente minimal si

	\begin{itemize}
		\item $T(X)$ es suficiente, y
		\item $\forall T'(X)$ estadístico suficiente, existe una función $f$ tal que $T(X) = f(T'(X)).$ 
	\end{itemize}
\end{definition}  





 \noindent \red{FALTA: Ejemplo estadístico minimal, particiones suficientes y comentarios sobre particiones} 

\clase{Clase: 22 de agosto}
Los estadísticos suficiente minimales están claramente definidos pero dicha definición no es útil para encontrar o construir  estadístico suficiente minimales. El siguiente Teorema establece una condición que permite evaluar si un estadístico es suficiente minimal 

\begin{theorem}[Suficiencia minimal]
	\label{teo:suficiencia_minimal}
	Sea $\familiaparametrica$ una familia dominada con densidades $\densidadparametrica$ y asuma que existe un estadístico $T(X)$ tal que para cada $x,y\in\cX$:
	\begin{equation}
		\frac{p_\theta(x)}{p_\theta(y)} \text{ no depende de }\theta \Leftrightarrow T(x) = T(y)
	\end{equation}
	entones, $T(X)$ es suficiente minimal.
\end{theorem}

Antes de probar este teorema, veamos un ejemplo aplicado a la distribución de Poisson. 
\begin{example}
	Recordemos que la distribución de Poisson (de parámetro $\theta$) modela la cantidad de eventos en un intervalo de tiempo de la forma y consideremos las observaciones $x=(x_1,\ldots, x_n)\sim Poisson(\theta)$ con verosimilitud
	\begin{equation}
		p_\theta(x) = \prod_{i=1}^n\frac{e^{-\theta}\theta^{x_i}}{x_i!} = \frac{e^{-n\theta}\theta^{\sum_{i=1}^n x_i}}{\prod_{i=1}^nx_i!}
	\end{equation}
	Notemos que la razón de verosimilitudes para dos observaciones $x,y\in\cX$ toma la forma 
		\begin{equation}
		\frac{p_\theta(x)}{p_\theta(y)} = \frac{\theta^{\sum_{i=1}^n x_i - \sum_{i=1}^n y_i}} {\prod_{i=1}^nx_i! / \prod_{i=1}^ny_i!}= 
	\end{equation}
	lo cual no depende de $\theta$ únicamente si $\sum_{i=1}^n x_i  = \sum_{i=1}^n y_i$, consecuentemente, $T(x) = \sum_{i=1}^n x_i$ es un estadístico suficiente de acuerdo al Teorema \ref{teo:suficiencia_minimal}.
\end{example}

\begin{proof}[Demostración de Teorema \ref{teo:suficiencia_minimal}] Primero veremos que $T$ es suficiente. Dada la partición inducida por el estadístico $T(X)$, para un valor $x\in\cX$ consideremos $x_T\in\{x'; T(x') = T(x)\}$, entonces
\begin{equation}
 	p_\theta(x) = \underbrace{{p_\theta(x) }/{p_\theta(x_T) }}_{h(x)\text{ indep.  } \theta} \underbrace{p_\theta(x_T) }_{q_\theta(T(x))	}
 \end{equation} 
 donde la no dependencia de $\theta$ se tiene por el supuesto del Teorema. 

 Para probar que el estadístico es suficiente minimal, asumamos que existe otro estadístico $T'(X)$, consideremos dos valores en la misma clase de equivalencia, i.e., $x,y,\ \tq\ T'(x) = T'(y)$, y veamos que  (mediante el CFNF) podemos escribir la razón de verosimilitudes de la forma

 \begin{equation}
 	\frac{p_\theta(x)}{p_\theta(y)} = \frac{g'_\theta(T'(x))h'(x)}{g'_\theta(T'(y))h'(y)} = \frac{h'(x)}{h'(y)}, \quad \text{pues } T'(x) = T'(y) 
 \end{equation}
 consecuentemente, el enunciado nos permite aseverar que como $\frac{p_\theta(x)}{p_\theta(y)}$ no depende de $\theta$, entonces $T(x) = T(y)$. Es decir, hemos mostrado que $T'(x) = T'(y)$ implica $T(x) = T(y)$, por lo que $T$ es función de $T'$.
	
\end{proof}

Como hemos discutido durante este capítulo, un objetivo principal de construir y estudiar estadísticos es su rol en el diseño y las propiedades de los estimadores. La noción de \textit{completitud} es clave en esta tarea. 

\begin{definition}[Estadístico completo]
	Un estadístico $T(X)$ es completo si para toda función $g$, se tiene que 
	\begin{equation}
		\E{g(T)|\theta} = 0, \forall \theta\in\Theta \Rightarrow Pr(g(T)=0) = 1
	\end{equation}
	
\end{definition}
El concepto de completitud dice relación con la construcción de estimadores usando estadísticos, lo cual puede ser ilustrado mediante el siguiente ejemplo

\begin{example}
	Consideremos dos estimadores, $\phi_1, \phi_2$ insesgados de $\theta$ distintos, es decir, 
	\begin{equation}
	\E{\phi_1} = \E{\phi_2} = \theta, \ \Prob{\phi_1\neq \phi_2} > 0
	\end{equation}
	Definamos ahora $\phi = \phi_1 - \phi_2$, donde verificamos que $\E{\phi} = 0, \forall \theta$, es decir, $\phi$ es un estimador insesgado de cero. Sin embargo, del supuesto anterior tenemos que $\Prob{\phi_1 - \phi_2=0}>0$, por lo que de acuerdo a la definición anterior, el estadístico $\phi$ no es completo. 
\end{example}
Intuitivamente entonces, podemos entender la noción de completitud como lo siguiente: un estadístico es completo si la única forma de construir un estimador insesgado de cero a partir de él es aplicándole la función idénticamente nula.  Veamos un ejemplo de la distribución Bernoulli, donde el estadístico $T(x) = \sum x_i$ es efectivamente completo. 

\begin{example}
	Sea $x=(x_1,\ldots,x_n)$ observaciones de $X\sim\ber{\theta}$, recordemos que $T(x) = \sum x_i\sim\bin{n,\theta}$, por lo que la esperanza $g(T)$ está dada por
	\begin{equation}
		\Et{g(T)} = \sum_{t=0}^n g(t)\binom{n}{t}\theta^t(1-\theta)^{n-t}= (1-\theta)^n\sum_{t=0}^n g(t)\binom{n}{t}\left(\frac{\theta}{1-\theta}\right)^t
	\end{equation}
	es decir un polinomio de grado $t$ en $r=\theta/(1-\theta)\in\R_+$, entonces, $\Et{g(T)} = 0$ implica que necesariamente los pesos de este polinomio sean todos idénticamente nulos: ${g(T)} = 0$
\end{example}


\section{La familia exponencial}

Hasta este punto, hemos considerado algunas distribuciones paramétricas, tales como Bernoulli, Gaussiana o Poisson, para ilustrar distintas propiedades y definiciones de los estadísticos. En esta sección, veremos que realmente todas estas distribuciones (y otras más) pueden escribirse de forma unificada. Para esto, consideremos la siguiente expresión llamada \textit{log-normalizador} (la razón de este nombre será clarificada en breve).
\begin{equation}
	\label{eq:lognormalizador}
	A(\eta) = \log\int_\cX \expo{\sum_{i=1}^s\eta_iT_i(x)}h(x)\dx
\end{equation}
donde definimos lo siguiente:
\begin{itemize}
	\item $\eta = [\eta_1,\ldots,\eta_s]^\top$ es el parámetro natural
	\item $T = [T_1,\ldots,T_s]^\top$ es un estadístico
	\item $h(x)$ es una función no-negativa
\end{itemize}
Definamos la siguiente función de densidad de probabilidad parametrizada por $\eta\in\{\eta | A(\eta)<\infty\}$
\begin{equation}
	\label{eq:densidadexponencial}
 	p_\eta(x) = \expo{\sum_{i=1}^s\eta_iT_i(x)-A(\eta)}h(x)
 \end{equation} 
 donde el hecho que $p_\eta(x)$ integra uno puede claramente verificarse reemplazando la ecuación \eqref{eq:lognormalizador} en \eqref{eq:densidadexponencial}, con lo cual se puede ver que $A$ definido en \eqref{eq:lognormalizador} es precisamente el logaritmo de la constante de normalización de la densidad definida en \eqref{eq:densidadexponencial}.
\clase{clase 27/8}
 Notemos que el estadístico $T$ es en efecto une estadístico suficiente para $\nu$ en la familia exponencial. En efecto, notemos que 
 \begin{equation}
	\label{eq:densidadexponencial2}
 	p_\eta(x) = \underbrace{\expo{\sum_{i=1}^s\eta_iT_i(x)-A(\eta)}}_{g_\theta(T(x))}\underbrace{h(x)}_{h(x)}
 \end{equation} 
consecuentemente, por el CFNF en el Teorema \ref{teo:neyman-fisher}, tenemos que $T$ es 
un estadístico suficiente para $\nu$.

Muchas de las distribuciones que usalmente consideramos pertenecen a la familia exponencial, por ejemplo, la distribución normal, exponencial, gamma, chi-cuadrado, beta, Dirichlet, Bernoulli, categórica, Poisson, Wishart (inversa) y geométrica. Otras distribuciones solo pertenecen a la familia exponencial para una determinada elección de sus parámetros, como lo ilustra el siguiente ejemplo.

\begin{example}[El modelo binomial pertenece a la familia exponencial]
Recordemos la distribución binomial está dada por 

\begin{align*}
	\bin{x|\theta,n} 	&=\binom{n}{x}\theta^x(1-\theta)^{n-x},\quad x\in\{0,1,2\ldots,n\}\\
					&=
					\underbrace{\binom{n}{x}}_{h(x)}\expo{x\underbrace{\loga{\frac{\theta}{1-\theta}}}_{\text{parámetro natural}} + \underbrace{n\loga{1-\theta}}_{-{A(\theta)}}}\\
\end{align*}
consecuentemente, para que $h(x)$ sea únicamente una función de la variable aleatoria, entonces el número de intentos $n$ tiene que se una cantidad conocida, \textbf{no un parámetro}. 	
 \end{example} 


\red{Falta: dar ejemplos de cómo las distribuciones conocidas (Bernoulli, Gaussian,Poisson, etc) se pueden generar desde la ecuación \eqref{eq:densidadexponencial}}

La familia exponencial va a ser ampliamente usada durante el curso, lo cual se debe a sus propiedades favorables para el análisis estadístico. Por ejemplo, el producto de dos distribuciones de la familia exponencial también pertenece a la familia exponencial. En efecto, consideremos dos VA $X_1,X_2,$ con distribuciones en la familia exponencial respectivamente dadas por
\begin{align}
	p_1(x_1) &= h_1(x_1)\expo{\theta_1T_1(x_1)-{A_1(\theta_1)}}\\
	p_2(x_2) &= h_2(x_2)\expo{\theta_2T_2(x_2)-{A_2(\theta_2)}}
\end{align}
si asumimos que estas VA son independientes, entonces densidad conjunta de $X=(X_1,X_2)\sim p$ está dada por
\begin{align}
	p(X) 	&= p_1(x_1) p_2(x_2) \nonumber \\ 
			&= \underbrace{h_1(x_1)h_2(x_2)}_{h(x)}\expo{\underbrace{[\theta_1,\theta_2]}_{\theta}\underbrace{
			\left[ \begin{array}{c}
			T_1(x_1)  \\
			T_2(x_2)  \end{array} \right]
			}_{T(x)} - \underbrace{\left(A_1(\theta_1)+A_2(\theta_2) \right)}_{A(\theta)}} 
\end{align}
con lo que eligiendo $\theta=[\theta_1,\theta_2]$ y $T=[T_1,T_2]$, vemos que $X$ está dado por una distribución de la familia exponencial.  

Otra propiedad de las familia exponencial es la relación entre los momentos de la distribucion y el lognormalizador $A$. Denotando
\begin{equation}
	Q(\theta) =\expo{A(\theta)} = \int_\cX \expo{\theta T (x)}h(x)\dx
\end{equation}
Observemos que la derivada de $A(\theta)$ está dada por 
\begin{align}
	\frac{dA(\theta)}{d\theta} &= Q^{-1}(\theta)\frac{d Q(\theta)}{d\theta} \\ 
	&= \frac{\int_\cX T (x) \expo{\theta T (x)}h(x)\dx}{\int_\cX \expo{\theta T (x)}h(x)\dx} \nonumber\\
	&= \frac{\int_\cX T (x) \expo{\theta T (x) -A(\theta)}h(x)\dx}{\int_\cX \expo{\theta T (x)-A(\theta) }h(x)\dx} \quad\quad \cdot  A(\theta)/A(\theta) \nonumber\\
	&=\E{T(x)}  \nonumber
\end{align}


