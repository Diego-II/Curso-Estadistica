%preamble para fuente, creo que Palatino es mejor para libros (FT)
\usepackage{mathpazo} % math & rm
\linespread{1.05}        % Palatino needs more leading (space between lines)
\usepackage[scaled]{helvet} % ss
\usepackage{courier} % tt
\normalfont
\usepackage[T1]{fontenc}

%Libro es en español (FT)
\usepackage[utf8]{inputenc}
\usepackage[spanish]{babel}

%tamaño del libro con marcas para reducir el tamaño carte (FT)
\usepackage{geometry}
%\geometry{layoutheight=230mm,layoutwidth=160mm,layoutvoffset=30mm,layouthoffset=20mm,showcrop}


\usepackage{cancel}

\usepackage{amsmath}

 

%theorems
\usepackage{amsthm}
\newtheorem{definition}{Definición}[section]
\newtheorem{example}{Ejemplo}[section]
\newtheorem{exercise}{Ejercicio}[section]
\newtheorem{theorem}{Teorema}[section]
\newtheorem{remark}{Observación}[section]

%macros
\newcommand*\clase[1]{\vspace{2em}{\color{red}\par\noindent\raisebox{.8ex}{\makebox[\linewidth]{\hrulefill\hspace{1ex}\raisebox{-.8ex}{#1}\hspace{1ex}\hrulefill}}\vspace{0em}}}

\usepackage{color}
\providecommand{\red}[1]{\textcolor{red}{{\bf #1}}}




\def\familiaparametrica{\mathcal{P}  = \{P_\theta\ \tq\ \theta\in\Theta\}}
\def\densidadparametrica{\{p_\theta\ \tq\ \theta\in\Theta\}}
\newcommand{\E}[1]{\mathbb{E} \left(#1\right)}
\newcommand{\V}[1]{\mathbb{V} \left(#1\right)}
\newcommand{\Prob}[1]{\mathbb{P}_\theta \left(#1\right)}
\newcommand{\Et}[1]{\mathbb{E}_\theta \left(#1\right)}
\newcommand{\Vt}[1]{\mathbb{V}_\theta \left(#1\right)}
\newcommand{\Probt}[1]{\mathbb{P} \left(#1\right)}
\newcommand{\ber}[1]{\text{Ber}\left(#1\right)}
\newcommand{\bin}[1]{\text{Bin}\left(#1\right)}
\newcommand{\uni}[1]{\text{Uniforme}\left(#1\right)}
\newcommand{\poi}[1]{\text{Poisson}\left(#1\right)}
\newcommand{\expo}[1]{\exp\left(#1\right)}
\newcommand{\loga}[1]{\log\left(#1\right)}
\newcommand{\KL}[2]{\text{KL}\left(#1\middle\|#2\right)}

%conjuntos
\def\R{{\mathbb R}}

%simbolos
\def\cP{{\mathcal P}}
\def\cA{{\mathcal A}}
\def\cB{{\mathcal B}}
\def\tq{{\text{t.q.}}}
\def\cX{{\mathcal X}}
\def\cY{{\mathcal Y}}
\def\cN{{\mathcal N}}
\def\cT{{\mathcal T}}
\def\thetaMV{{\theta_\text{MV}}}
\def\thetahat{{\hat\theta}}
\def\xb{{ \bar{x}}}
\def\gh{{ \hat{g}}}
\def\ghX{{ \hat{g}(X)}}
\def\ghx{{ \hat{g}(x)}}



\def\d{{\text{d}}}
\def\dx{{\text{d}x}}
\def\ind{{\mathbb 1}}


