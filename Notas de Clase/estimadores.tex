%!TEX root = apunte_estadistica.tex


\chapter{Estimadores}

Consideremos una función del parámetro de una familia paramétrica $\familiaparametrica$, $g(\theta)$.  Un estimador puntual de $g(\theta)$ es un estadístico, es decir, una función de la VA $X$, que toma valores en el mismo conjunto que $g(\Theta)$. En general denotaremos como $\gh(X)$ el estimador de $g(\theta)$ aplicado a $X$ 


\begin{example}[Estimador de la media Gaussiana]
	\label{ex:estimador_media}
	Consideremos $X = (X_1,\ldots,X_n)\sim\cN(\mu,\sigma^2)$. Un estimador de $g(\theta) = g(\mu,\sigma) = \mu$ es el estadístico 
	\begin{equation}
		\gh(X) = \frac{1}{n}\sum_{i=1}^nX_i
	\end{equation} 
\end{example}

Una clase muy importante de estimadores son los estimadores insesgados. 

\begin{definition}[Estimador insesgado]
	\label{def:estimador_insesgado}
	Sea $\ghx$ un estimador de $g(\theta)$. Este estimador es insesgado si 
	\begin{equation}
		\E{\ghx} = g(\theta)
	\end{equation}
	y el sesgo de $\gh$ se define como 
	\begin{equation}
		b_\gh(\theta) = \E{\ghx} - g(\theta)
	\end{equation}
\end{definition}



Los estimadores insesgados juegan un rol importante en el estudio y aplicación de la estadística, sin embargo, uno no siempre debe poner exclusiva atención a ellos. Los siguiente ejemplos ilustran el rol del estimador insesgado en dos familias paramétricas distintas. 

\begin{example}[Estimador insesgado de la media Gaussiana]
	\label{ex:estimador_in_media}
	El estimador de $g(\theta) =  \mu$ descrito en el Ejemplo \ref{ex:estimador_media} es insesgado, en efecto: 
	\begin{equation}
		\E{\ghx} = \E{\frac{1}{n}\sum_{i=1}^nX_i}	= \frac{1}{n}\sum_{i=1}^n\E{X_i}		= \frac{1}{n}\sum_{i=1}^n\mu = \mu	
	\end{equation}
\end{example}


\begin{example}[Estimador de la taza de la distribución exponencial\footnote{Schervish}]
	\label{ex:estimador_exponancial}
	Consideremos $X\sim Exp(\theta)$, donde $Exp(x|\theta) = \theta\exp(-\theta x)$, y asumamos que existe un estimador insesgado $\ghX$ de $g(\theta) = \theta$, entonces, 
	\begin{equation}
		\E{\ghX} = \int_0^\infty \ghx\theta\exp(-\theta x)\d x = \theta, \forall \theta,
	\end{equation}
	lo cual es equivalente a decir que $\int_0^\infty \ghx\exp(-\theta x)\d x = 1, \forall \theta$ o bien que (al derivar ambos lados de esta expresión c.r.a. $\theta$)  $\int_0^\infty x\ghx\exp(-\theta x)\d x = 0, \forall \theta$.

	Esta última expresión es equivalente a que $\E{X\ghX} = 0$, lo que a su vez y considerando que $X$ es un estadístico suficiente y completo, implica que necesariamente la función $X\ghX=0$ c.s. $\forall \theta$, y también que $\ghX=0$ c.s. $\forall \theta$. Como esto contradice el hecho de que $\ghX$ es insesgado, no es posible construir estimadores insesgados para $\theta$ en la distribución exponencial.
\end{example}


Veamos ahora un ejemplo de un estimador sesgado de la varianza y cómo se puede construir un estimador insesgado. 

\begin{example}
Consideremos una familia paramétrica $\familiaparametrica$ y denotemos por $\mu$ y $\sigma^2$ su media y su varianza respectivamente. Usando las observaciones $x_1,x_2,\ldots,x_n$, calculemos la varianza del estimador de la media, dado por $\xb = \frac{1}{n}\sum_{i=1}^n x_i$ mediante
\begin{equation}
	\label{eq:varianza_media_muestral}
 	\Vt{\xb} = \Vt{\frac{1}{n}	\sum_{i=1}^n x_i}  \underbrace{=}_{\text{i.i.d.}}  \frac{1}{n^2}	\sum_{i=1}^n\Vt{ x_i} =\frac{\sigma^2}{n}
 \end{equation} 
 es decir, el estimador de la media usando $n$ muestras, tiene una varianza $\sigma^2/n$.

 Consideremos ahora el siguiente estimador para la varianza: 
\begin{equation}
	\label{eq:est_varianza_sesgado}
	S_2 = \frac{1}{n}\sum_{i=1}^n (x_i-\xb)^2
\end{equation}
y notemos que la esperanza de dicho estimador es
\begin{align}
	\label{eq:sesgo_varianza}
	\Et{S_2 } &= \Et{\frac{1}{n}\sum_{i=1}^n (x_i-\mu+\mu-\xb)^2}\nonumber\\
				&= \Et{ \frac{1}{n}\sum_{i=1}^n(x_i-\mu)^2 + 2\frac{1}{n}\sum_{i=1}^n(x_i-\mu)(\mu-\xb) + \frac{1}{n}\sum_{i=1}^n(\mu-\xb)^2}\nonumber\\
				&= \Et{ \frac{1}{n}\sum_{i=1}^n(x_i-\mu)^2 - 2(\mu-\xb)^2 + (\mu-\xb)^2}\nonumber\\
				&= \Et{ \frac{1}{n}\sum_{i=1}^n(x_i-\mu)^2 - (\mu-\xb)^2}\nonumber\\
				&= \Vt{x_i} - \Vt{\xb}\quad\text{ver ecuación \eqref{eq:varianza_media_muestral}}\nonumber\\
				&= 	\sigma^2 + \sigma^2/n = \left(\frac{n+1}{n}\right)\sigma^2
\end{align}
Esto quiere decir que el sesgo del estimado en la ecuación \eqref{eq:est_varianza_sesgado} es asintóticamente insesgado, es decir, que su sesgo tiende a cero cuando el número de muestas $n$ tiende a infinito. Sin embargo, notemos que podemos corregir el estimado de la varianza multiplicando el estimador original, $S_2$ en la ecuación \eqref{eq:est_varianza_sesgado} por $n/(n+1)$, con lo que el estimador corregido denotado por 
\begin{equation}
	\label{eq:est_varianza_insesgado}
	S'_2 = \frac{n}{n+1}S_2 =  \frac{1}{n+1}\sum_{i=1}^n (x_i-\xb)^2
\end{equation}
cumple con
\begin{equation}
	\Et{S'_2 } =  \left(\frac{n}{n+1}\right)\Et{S_2} \underbrace{=}_{\text{ec.}\eqref{eq:sesgo_varianza}} \left(\frac{n}{n+1}\right) \left(\frac{n+1}{n}\right)\sigma^2 = \sigma^2
\end{equation}
es decir, el estimador $S'_2$ en la ecuación \eqref{eq:est_varianza_insesgado} es insesgado.
\end{example}

\clase{Clase 8: 29/8}
Para tener una notación más limpia, desde ahora nos referiremos a estimadores $\phi=\gh$ de $\theta$ en general para evitar la expresión más engorrosa estimador $\gh(X)$ de $g(\theta)$.



Es natural evaluar la bondad de distintos estimadores (sesgados o insesgados), una forma de hacer esto es definir una función de \textit{pérdida} o \textit{costo} que compara el valor reportado por el estimador y el valor real del parámetro. En general, elegimos la función de pérdida cuadrática para un estimador $\phi$ y un parámetro $\theta$ definida por 
\begin{equation}
	L_2(\phi,\theta)^2 = (\phi - \theta)^2.
\end{equation}
Luego, como el estimador es una VA, también lo es la función de pérdida, por lo que podemos calcular la esperanza de la función de pérdida, lo cual conocemos como \textit{riesgo}. El riesgo asociado a la pérdida cuadrática en la ecuación anterior está dado por: 
\begin{alignat}{3}
 	R(\theta, \hat{g})  &= \E{(\theta - \phi)^2}\nonumber\\
 						& = \E{\left(\theta - \bar{\phi}+ \bar{\phi} -\phi\right)^2}; \quad \text{denotando }\bar{\phi} = \E{ \phi}\nonumber\\
 						& = \E{(\theta - \bar{\phi})^2+2(\theta - \bar{\phi})\cancel{(\bar{\phi} -\phi)} +  (\bar{\phi} -\phi)^2}\nonumber\\
 						& = \underbrace{(\theta - \bar{\phi})^2}_{=b_{\phi}^2\ (\text{sesgo}^2)} +  \underbrace{\E{(\bar{\phi} -\phi)^2}}_{=V_{\phi}\ \text{(varianza)}}.\label{eq:riesgo_cuad}
 \end{alignat} 


Con esta métrica para comparar estimadores, el siguiente teorema establece que la información reportada por un estadístico suficiente (Definición \ref{def:estadístico_suficiente}), puede solo mejorar un estimador. 

\begin{theorem}[Teorema de Rao-Blackwell]
	\label{teo:rao-blackwell}
	Sea $\phi = \phi(X)$ un estimador de $\theta$ tal que $\Et{\phi}<\infty, \forall \theta$. Asumamos que existe $T$ estadístico suficiente para $\theta$ y sea $\phi^\star = \Et{\phi|T}$. Entonces, 
	\begin{equation}
		\Et{(\phi^\star-\theta)^2} \leq \Et{(\phi-\theta)^2}, \forall\theta,
	\end{equation}
	donde la desigualdad es estricta salvo en el caso donde $\phi$ es función de $T$.
\end{theorem}

En otras palabras, el Teo. de Rao-Blackwell establece que un estimador puede ser \textit{mejorado} si es reemplazado por su esperanza condicional dado un estadístico suficiente. El proceso de mejorar un estimador poco eficiente de esta forma es conocido como \textit{Rao-Blackwellización} y veremos un ejemplo a continuación.


\begin{example}
Consideremos $X = (X_1,\ldots,X_n)\sim \poi{\theta}$ y estimemos el parámetro $\theta$. Para esto, consideremos el estimador básico $\phi = X_1$ y \textit{Rao-Blackwellicémoslo} usando el estimador suficiente $T=\sum_{i=1}^nX_i$, es decir, 
\begin{equation}
	\phi^* = \Et{X_1\middle|\sum_i X_i=t}.
\end{equation}
Para calcular esta esperanza condicional, observemos primero que  
\begin{equation}
	\sum_{j=1}^n\Et{X_j\middle|\sum_{i=1}^n X_i=t} = \Et{\sum_{j=1}^nX_j\middle|\sum_{i=1}^n X_i=t} = t
\end{equation}
y que como $X_1,\ldots,X_n$ son iid, entonces todos los términos dentro de la suma del lado izquierdo de la ecuación anterior son iguales. Consecuentemente, recuperamos el estimador
\begin{equation}
 	\phi^* = \frac{t}{n} = \frac{1}{n}\sum_{i=1}^nX_i
 \end{equation} 
\end{example}

Para demostrar el Teorema \ref{teo:rao-blackwell} consideremos dos variable aleatorias $X\in\cX$, $Y\in\cY$, y recordemos dos propiedades básicas. En primer lugar la ley de esperanzas totales, la cual establece que 
\begin{alignat}{3}
	\mathbb{E}_Y{\mathbb{E}_{X|Y}{(X|Y)}} &= \int_\cY\int_\cX x \d P(x|y) \d P(y) \quad\quad&&\text{def. esperanza}\nonumber\\
				&=  \int_\cX x \int_\cY \d P(x|y) \d P(y) &&\text{linealidad}\nonumber\\
				&=  \int_\cX x \int_\cY \d P(x,y) &&\text{def. esperanza condicional}\nonumber\\
				&=  \int_\cX x \d P(x) = \mathbb{E}_X(X) &&\text{def. esperanza} \label{eq:total_expectation}
\end{alignat}
y la desigualdad de Jensen, la cual para el caso particular del costo cuadrático, puede verificarse que
\begin{equation}
	0 \leq \V{X} =  \E{X^2}-\E{X}^2 \Rightarrow \E{X^2} \geq \E{X}^2. \label{eq:jensen_var}
\end{equation}
\red{Falta: dibujo con la intuición detrás de Jensen en el caso general}

Entonces, utilizando las expresiones en \eqref{eq:total_expectation} y \eqref{eq:jensen_var}, podemos demostrar el teorema anterior.

 \begin{proof}[Demostración de Teorema \ref{teo:rao-blackwell}]
 	La varianza del estimador $\phi^\star$ está dada por 
 	\begin{alignat*}{2}
 		\Et{(\phi^\star-\theta)^2} &= \Et{(\Et{\phi|T}-\theta)^2} \quad\quad\quad &&\text{def.}\\
 								&= \Et{(\Et{\phi-\theta|T})^2}&& \text{linealidad}\\
 								&\leq \Et{\Et{(\phi-\theta)^2|T}}&& \text{Jensen}\\
 								&= \Et{(\phi-\theta)^2} &&\text{ley esperanzas totales}
 	\end{alignat*}
Donde las esperanzas exteriores son con respecto a $T$ y las interiores con respecto a $X$ (o equivalentemente a $\phi$).  Observemos además que la desigualdad anterior viene de la expresión en la ecuación \eqref{eq:jensen_var}, por lo que la igualdad es obtenida si $\V{\phi-\theta|T} = 0$, es decir, la VA $\phi-\theta$ tiene que ser constante para cada valor de $T$, es decir, $\phi$ es función de $T$. Intuitivamente podemos entender esto como que si el estadístico ya fue considerado en el estimador, entonces conocer el valor del estadístico no reporta información adicional. 
 \end{proof}

\begin{remark}
	Notemos que si el estimador $\phi$ es insesgado, su \textit{Rao-Blackwellización} $\phi^*$ también lo es, en efecto
	\begin{equation}
		\Et{\phi^*} = \Et{\Et{\phi|T}} = \Et{\phi} = \theta,
	\end{equation}
	donde la segunda igualdad está dada por la ley de esperanzas totales y la tercera por el supuesto de que $\phi$ es insesgado.
\end{remark}



En base al riesgo cuadrático definido en la ecuación \eqref{eq:riesgo_cuad}, podemos ver que si un estimador es insesgado (Definición \ref{def:estimador_insesgado}), su riesgo cuadrático es únicamente su varianza. Esto motiva la siguiente definición de optimalidad para estimadores insesgados. 

 \begin{definition}[Estimador insesgado de varianza uniformemente mínima]
  	El estimador $\phi$ de $\theta$ es un estimador insesgado de varianza uniformemente mínima (EIVUM) si es insesgado y si $\forall \phi':\cX\rightarrow \Theta$ estimador insesgado se tiene
  	\begin{equation}
  		\Vt{\phi}\leq\Vt{\phi'}, \forall \theta\in\Theta.
  	\end{equation}
  \end{definition} 

\begin{example}
	Consideremos $x=(x_1,\ldots,x_n)\sim\ber{\theta}$ y los siguientes estimadores de $\theta$
	\begin{itemize}
		\item $\phi_1(x) = x_1$
		\item $\phi_2(x) = \frac{1}{2}(x_1+x_2)$
		\item $\phi_3(x) = \frac{1}{n}\sum_{i=1}^n x_i$
	\end{itemize}
	Observemos que todos estos estimadores son insesgados, pues como $\forall i, \Et{x_i} = \theta$, entonces 
	\begin{equation}
		\Et{\phi_1(x)} = \Et{\phi_2(x)} = \Et{\phi_3(x)} = \theta
	\end{equation}
	Veamos ahora que la varianza de $\phi_3(x)$ está dada por
	\begin{equation}
		\Vt{\phi_3(x)} = \Vt{\frac{1}{n}\sum_{i=1}^n x_i} = \frac{1}{n^2}\sum_{i=1}^n \Vt{x_i} = \frac{\theta(1-\theta)}{n}
	\end{equation}
	pues $\Vt{x_i} = \Et{(\theta - x_i)} = \Et{x_i^2} - \theta^2 = (0^2 \cdot (1-\theta) + 1^2 \cdot \theta) - \theta^2 = \theta(1-\theta)$. Consecuentemente, la varianza de los estimadores considerados decae como la inversa del número de muestras.
\end{example}

Con las definiciones anteriores, podemos mencionar el siguiente teorema, el cual conecta la noción de estadístico completo con la de EIVUM. 

\begin{theorem}[Teorema de Lehmann-Scheffé]
	Sea $X$ una VA con distribución paramétrica $\familiaparametrica$ y $T$ un estadístico suficiente y completo para $\theta$. Si el estimador $\phi = \phi(T)$ de $\theta$ es insesgado, entonces $\phi$ es el único EIVUM. 
 \end{theorem} 

 \begin{proof}
 	Veamos en primer lugar que es posible construir un estimador en función del estadístico $\phi(T)$ que tiene menor o igual varianza que un estimador arbitrario $\phi'(X)$. En efecto, el Teorema de Rao-Blackwell estable ce que el estimador 
 	\begin{equation}
 		\phi(T) = \Et{\phi'(X)|T},
 	\end{equation}
 	tiene efectivamente menos varianza que $\phi'(X)$.

 	Ahora veamos que solo existe un único estimador insesgado que es función de $T$, en efecto, si existiesen dos estimadores insesgados de $\theta$, $\phi_1(T),\phi_2(T)$, entonces, $\Et{\phi_1(T)-\phi_1(T)}=0$ y como $T$ es completo, entonces, $\phi_1(T) = \phi_2(T)$ c.s.-$P_\theta$.

 	Hemos probado que (i) para un estimador arbitrario, se puede construir un estimador que es función de $T$ el cual tiene menor o igual varianza que el estimador original y, (ii) el estimador insesgado $\phi(T)$ es único. Consecuentemente, $\phi(T)$ es el único EIVUM.
 \end{proof}

El Teorema de Lehmann-Scheffé da una receta para encontrar el EIVUM: simplemente es necesario encontrar un estadístico completo y construir un estimador insesgado en base a éste, esto garantiza que el estimador construido es el \textbf{único} EIVUM.
\begin{example}[EIVUM para Bernoulli]
	Recordemos que en el Ejemplo \ref{eq:est_completo_bernoulli} vimos que el estadístico $T=\sum_{i=1}^nX_i$ es completo para $X\sim\ber{\theta}$. Como el estimador de $\theta$ dado por $\phi(T) = T/n$ es insesgado, 
\begin{equation}
	\Et{\phi(T)} = \Et{T/n} = \sum_{i=1}^n \Et{X_i} /n = \theta,
\end{equation}
entonces $\phi(T) = T/n$ es el EIVUM para $\theta$ en $\ber{\theta}$.	
\end{example}

	
